\documentclass[10pt]{relatorio_tcc_ads_ifba}
\begin{document}

% Aluno/autor do documento (obrigatório)
% e.g. \aluno{Luiz Felipe Torres Farias}
\aluno{Fulano de tal}

% Titulo do seu projeto (obrigatório)
% e.g. \titulo{Desenvolvimento de Software XPTO}
\titulo{Titulo do Seu Projeto}

% Data da sua defesa (obrigatório)
% e.g. \date{21 de janeiro de 2024}
\date{21 de janeiro de 2024}

% Orientador (obrigatório)
% Opção: [f] para orientadora. O valor default é [m]
% e.g. \orientador[f]{Flavia Maristela}
% e.g. \orientador{Frederico Barboza}
\orientador[f]{Ciclana de tal}

% Coorientador (opcional)
% Opção: [f] para coorientadora. O valor default é [m]. Caso não possua coorientador, comentar ou deletar essa linha.
% e.g. \orientador[f]{Flavia Maristela}
% e.g. \orientador{Frederico Barboza}
% \coorientador{NOME}
\coorientador{Beltrano de tal}

% Pretextual (obrigatório)
% Comando responsável por imprimir o conteúdo pré-textual (capa e sumário)
\pretextual

% ---------------------------------------------------------------------
% INÍCIO DA PARTE ESCRITA 
% ---------------------------------------------------------------------
\section{Visão geral }

% Recomendo manter cada seção ou subseção de texto em um arquivo separado e depois utilizados com o comando include, conforme o exemplo a seguir. Porém, se preferir, também pode deixar seu texto neste arquivo
\subsection{Objetivo}
Lorem ipsum dolor sit amet, consectetur adipiscing elit. Sed cursus rutrum consequat. Suspendisse sit amet posuere ante. Integer sed turpis est. Ut elementum dignissim nibh. Nulla faucibus massa eget vulputate interdum. Fusce orci ligula, gravida consectetur risus ut, elementum tristique tellus. Quisque viverra auctor lorem, sed pharetra sapien vulputate eu. Sed vestibulum, turpis non sagittis volutpat, ipsum dolor finibus tortor, sed placerat risus lacus id lorem. \cite{c_freitas_uma_2014}
\subsection{Definições, Siglas e Abreviações}
Duis euismod orci nec laoreet suscipit. Nunc et euismod odio. Aenean vulputate tortor mollis, porta erat id, sodales odio. Aenean congue risus quis ornare elementum. Aliquam erat volutpat. Integer non elit id turpis dignissim hendrerit in eget justo. Praesent libero tortor, porttitor sed urna non, ultricies vehicula mi. Pellentesque sit amet porta lectus. Donec dapibus libero at volutpat placerat. Integer turpis diam, euismod non eros ut, aliquam lobortis diam. Nullam id imperdiet turpis. Proin egestas justo ac elit sodales, at aliquam enim efficitur. \cite{da_cruz_marques_reserva_2014}

\subsection{Declaração do Problema}
\textcolor{blue}{
    [Explique o problema que está sendo solucionado por este projeto. Descreva o contexto, o cenário do problema, explique porque resolveu trabalhar com ele no seu TCC. Descreva a motivação e Justificativa para trabalhar com este problema.]
} \\
\textcolor{red}{[Obrigatório]}

\subsection{Proposta de Solução de Software}
\textcolor{blue}{
    [Descreva a solução de software em nível conceitual que está sendo desenvolvida para solucionar o problema proposto por este projeto. Como seu software é importante na solução do problema.]
} \\
\textcolor{red}{[Obrigatório]}

\subsection{Tecnologias da solução}
\textcolor{blue}{
    [Descreva as tecnologias a serem utilizadas no desenvolvimento da solução de software que está sendo desenvolvida para solucionar o problema proposto por este projeto. Quais softwares, plataformas, ferramentas, linguagens, etc. Justifique suas escolhas.]
} \\
\textcolor{red}{[Obrigatório]}

\section{Requisitos}

\subsection{Requisitos Funcionais}
\textcolor{blue}{
    [Descreva os todos requisitos funcionais do sistema a ser desenvolvido. Utilize o formato de Histórias de Usuários. Por exemplo: Como um gerente de vendas, eu preciso saber quantos clientes minha loja atendeu hoje para eu possa redimensionar estratégias de vendas.]
} \\
\textcolor{red}{[Obrigatório]}

\subsection{Requisitos Não-Funcionais}
\textcolor{blue}{
    Descreva os requisitos não-funcionais do sistema a ser desenvolvido. Utilize o formato de texto simples com identificação do requisito. Por exemplo: [RNF1] O sistema deve ser acessível nas plataformas web e mobile.]
} \\
\textcolor{red}{[Obrigatório]}

\section{Design}
\subsection{Projeto UML}
\textcolor{blue}{
    [Insira os seguintes Diagramas de UML para o seu projeto: 
    \begin{enumerate}
        \item Diagrama de Classe
        \item Diagrama de Atividades da principal atividade do sistema
        \item Diagrama de Sequência da principal atividade do sistema
        \item Diagrama de Estado do principal objeto do sistema
        \item Diagrama de Componentes
        \item Diagrama de Implantação
    \end{enumerate}
    OBS: Como cada sistema tem particularidades específicas, a escolha de qual a principal atividade e qual o principal objeto do sistema deve ser discutida com o orientador. Consideramos que existem alguns diagramas básicos que devem existir em qualquer projeto. São eles: Diagrama de Classes, Componentes e Implantação. A elaboração dos outros diagramas devem ser discutidos com seu orientador.]
} \\
\textcolor{red}{
    [Obrigatório - A seção é obrigatória, os diagramas de classes e diagrama de casos de uso são obrigatórios. Os outros diagramas são opcionais a depender do tipo do sistema a ser desenvolvido. Deve ser discutido com o orientador quais diagramas devem constar no trabalho.]
}

\subsection{Visão arquitetural}
\textcolor{blue}{
    [Descreva o estilo arquitetural que melhor se encaixa no seu sistema. Descreva a arquitetura de seu sistema, incluindo quais tecnologias utilizou na sua construção. Utilize figuras para ilustrar se necessário.]
} \\
\textcolor{red}{[Obrigatório]}

\subsection{Modelo de Banco de Dados}
\textcolor{blue}{
    [Descreva o modelo de Dados Físico e Lógico]
} \\
\textcolor{red}{
    [Obrigatório - Somente se o sistema possui banco de dados.]
}

\section{Qualidade}

\subsection{Projeto de testes}
\textcolor{blue}{
    [Descreva em detalhes as estratégias de Testes que utilizou para desenvolvimento do seu software. Se utilizou casos de testes, testes automatizados ou manuais, ferramentas, etc.]
} \\
\textcolor{red}{[Obrigatório]}

\section{Implantação}
\subsection{Projeto de implantação}
\textcolor{blue}{
    [Descreva a Plataforma de Hardware e Software requeridas para instalação e operação do seu software.]
} \\
\textcolor{red}{[Obrigatório]}

\section{Manual do Usuário}{
\textcolor{blue}{
    [Descreva o Manual de Usuário, como utilizar o seu software. Sugerimos que coloque figuras com as telas do seu sistema se necessário para melhorar o entendimento do usuário.]
} \\
\textcolor{red}{[Obrigatório]}
}

\section*{Agradecimentos}
\textcolor{blue}{
    [Agradecimentos a colaboradores do seu projeto]
} \\
\textcolor{red}{[Opcional]}

\bibliography{referencias}
\textcolor{blue}{
    [Forneça uma lista completa de todos os documentos mencionados ou que foram utilizados como referência na elaboração deste documento. Todos os documentos devem ser identificados por título, data, nome e organização responsável por sua publicação. Especifique as fontes dessas referências. Utilize o padrão ABNT para o formato das referências.]
} \\
\textcolor{red}{[Obrigatório]} \\
\textbf{[Dica: insira suas referencias no arquivo "referencias.bib" e utilize o comando "cite" para utilizá-la]}

\end{document}
